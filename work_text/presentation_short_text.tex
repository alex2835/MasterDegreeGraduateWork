\documentclass[a4paper,12pt]{diplom}
% \usepackage[latin1]{inputenc}
% \usepackage[utf8]{inputenc}
\inputencoding{utf8} % Кодировка вашего файла


\usepackage{paratype} % Шрифты (можно отключить, если дает ошибку)
%% Немного увеличим шрифт в математическом режиме, чтобы соответствовать размерам Paratype-шрифтов
\DeclareMathSizes{12}{13.4}{11}{10}

\usepackage[left=3cm,right=2cm,top=2cm,bottom=2cm]{geometry} % Размеры полей
\usepackage[onehalfspacing]{setspace} % Полуторный интервал
%\renewcommand{\baselinestretch}{1.25} % Полуторный интервал
\usepackage{indentfirst} % Абзацный отступ в начале разделов
\setlength{\parindent}{1.25cm} % Величина абзацного отступа

\usepackage[pdftex]{graphicx} % Для вставки изображений
\usepackage{array} % Для таблиц
\usepackage{booktabs} % Для красивых таблиц 
\usepackage{tikz} % Рисунки с помощью TikZ
\usepackage[linesnumbered,lined,ruled]{algorithm2e} % Для оформления псевдокода
%\usepackage{algorithm} % Альтернатива оформления псевдокода
%\usepackage{algpseudocode} % Альтернатива оформления псевдокода
\usepackage{listings} % Оформление листингов программ
\usepackage{icomma} % Удаляем тонкий пробел после запятой в мат. режиме
\usepackage{centernot}
\usepackage[figure,table]{totalcount}

\begin{document}



\section*{слайд 2  Задача обратной свертки}

Постановка основной задачи: Модификация и анализ возможностей методов восстановления обратной свертки распределений физических величин, 
в том числе на основе расширенного SVD-алгоритма.

При проведении физических экспериментов, интерес может представлять распределение измеряемой случайной величины. Однако, данные 
полученные в результате измерения, могут отличаться  истинных значений из-за возможных внешних факторов или конечного разрешения 
аппаратуры. 

Но нас интересует непосредственно, математическая задача, которая формулируется следующим образом. Есть множество пар предполагаемых 
истинных и измеренных значений, возможно многомерной величиной.

\section*{слайд 3 Гистограмный подход}
Гистограмный подход к решению данной задачи определяется следующим образом. Разбиваем множество значений на интервалы(бины), где 
каждый отражает количество попавших в него частиц.

Описание искажений определяется матрицей миграций, которая
определят вероятность попадания случайный величины в какой-то бин
при условии что истинное значение попадает в свой бин. Это дает нам
систему линейных отношений между смоделированным истинным и
измеренным распределениями


\section*{слайд 4 Матрица миграций}
Матрица миграций выглядит следующим образом


\section*{слайд 5 Задача минимизации}
Данную задачу предлагается решать используя регуляризационный подход. Представим задачу в виде задачи минимизации с добавлением 
слагаемого, которое хранит информацию об истинном распределении.

\section*{слайд 6 Матрица соседства}
Для одномерного случая в оригинальной работе предлагается брать следующее слагаемое, которое является суммой квадратов вторых 
производных, которое описаем гладкость искомого распределения.

\section*{слайд 7 Решение задачи минимизации}
Данная задача сводиться к решению переопределенной системе линейных уравнений. И решается с помощью сингулярного разложения матриц 
через последовательные замены.

\section*{слайд 8 регуляризационное слагаемое для многомерного случая}
При решении задачи обратной свертки с космическими частицами столкнулись с проблемой, неспособности стандартного подхода решать задачу в многомерном 
случаи. Так как гладкость учитывается только для соседних интервалов. То есть, например, для двумерно случая, гладкость относительно 
второго пространства не учитывается.

И в работе Юрия Викторовича была предложена следующая модификация, с использованием в качестве регуляризационного слагаемого матрицу Киргофа, 
которая в общем случаи описывает силу соседства. И при бинарном отношении(то есть если сосед то сила равна 1) в одномерном случаи как раз
является рассмотренной ранее матрицей C. 

И непосредственный интерес вызывало определения отношения соседства небинарным образом.


\section*{слайд 9 Варианты вычисления не бинарной матрицы соседства}
Для вычисления силы связи предлагалось рассмотреть следующие идеи. 

Определять на основе площади соприкосновения, чем больше площадь тем сильнее связь 

На основе расстояния между центрами масс, в обратной пропорции.

И последняя идея. Вычислять основываясь на количестве истинных значений попавших в этот бин. То есть чем элементов из истинного спектра 
попали в данный бин, тем сильнее связь между ними.


\section*{слайд 10 Биннинг}
В рамках работы встречались распределения следующего вида. В самых популярных пакетах ROOT И RooUnfold, в качестве дискретизации 
предлагается только разделение на равные интервалы. При таком разбиении, решение задачи обратной свертки, не имеет особого смысла, 
тк мы получаем, однин бин, минимально будет модифицирован и множество несостоятельных бинов, которые из-за малого объема информации 
приведут у еще большему росту погрешности.

\section*{слайд 11 Биннинг эвристики}
Раннее у меня возникла идея использования следующих эвристик\dots

Для начала были приняты попытки к формализации данной задачи и точному ее решению. Я попытался представить как задачу максимизации 
следующего вида. Рассмотрим пары истинных значений в виде отрезков лежащих на оси X, а бины представим прямы параллельных оси Y. 
Требуется расположить прямые таким образом чтобы количество пересечений было максимально, с условием некоторых ограничений, 
на возможное пересечение отрезка только один раз, и расстояния между бинами.

\end{document}