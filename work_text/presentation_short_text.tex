\documentclass[a4paper,12pt]{diplom}
% \usepackage[latin1]{inputenc}
% \usepackage[utf8]{inputenc}
\inputencoding{utf8} % Кодировка вашего файла


\usepackage{paratype} % Шрифты (можно отключить, если дает ошибку)
%% Немного увеличим шрифт в математическом режиме, чтобы соответствовать размерам Paratype-шрифтов
\DeclareMathSizes{12}{13.4}{11}{10}

\usepackage[left=3cm,right=2cm,top=2cm,bottom=2cm]{geometry} % Размеры полей
\usepackage[onehalfspacing]{setspace} % Полуторный интервал
%\renewcommand{\baselinestretch}{1.25} % Полуторный интервал
\usepackage{indentfirst} % Абзацный отступ в начале разделов
\setlength{\parindent}{1.25cm} % Величина абзацного отступа

\usepackage[pdftex]{graphicx} % Для вставки изображений
\usepackage{array} % Для таблиц
\usepackage{booktabs} % Для красивых таблиц 
\usepackage{tikz} % Рисунки с помощью TikZ
\usepackage[linesnumbered,lined,ruled]{algorithm2e} % Для оформления псевдокода
%\usepackage{algorithm} % Альтернатива оформления псевдокода
%\usepackage{algpseudocode} % Альтернатива оформления псевдокода
\usepackage{listings} % Оформление листингов программ
\usepackage{icomma} % Удаляем тонкий пробел после запятой в мат. режиме
\usepackage{centernot}
\usepackage[figure,table]{totalcount}

\begin{document}



\section*{слайд 2  Задача обратной свертки}
При проведении физических экспериментов, интерес может представлять распределение измеряемой случайной величины. Однако, данные 
полученные в результате измерения, могут отличаться  истинных значений из-за возможных внешних факторов или конечного разрешения 
аппаратуры.

Задачу обратной свертки можно сформулировать следующим образом. У нас есть представление о характере искажений, на основе которых 
можно сгенерировать пары истинных и измеренных значений, на основе которых мы хотим получить алгоритм, преобразующий, измеренное 
распределение в истинное.

\section*{слайд 3 Гистограмный подход}
Гистограмный подход к решению данной задачи определяется следующим образом. Разбиваем множество значений на интервалы(бины), где 
каждый отражает количество попавших в него частиц. Каждую запись в измеряемом бине можно напрямую проследить до ее происхождения. 
На основе этого можно определить систему линейных отношений между смоделированным истинным и измеренным распределениями.

\section*{слайд 4 Матрица миграций}
Данная система описывается матрицей миграций, которая строится следующим образом. Строка обозначает вероятность попадания истинного 
значений в i бин, при условии что измеренное попало в j бин.


\section*{слайд 5 Задача минимизации}

Данную задачу предлагается решить методом регулязации. Представим задачу в виде задачи минимизации с добавлением регулязационного 
слагаемого, которое хранит информацию об истинном распределении.

\section*{слайд 6 Матрица соседства}
Для одномерного случая в оригинальной работе предлагается брать следующее слагаемое, которое является суммой квадратов вторых 
производных, которое описаем гладкость искомого распределения.

\section*{слайд 7 Решение задачи минимизации}
Данная задача сводиться к решению переопределенной системе линейных уравнений. И решается с помощью сингулярного разложения матриц


\section*{слайд 8 Регулязационное слагаемое для многомерного случая}
При решении задачи с космическими частицами столкнулись с проблемой, неспособности стандартного подхода решать задачу в многомерном 
случаи. Например, при описании одной частицы, параметрами: жесткостью, полярным и азимутным углом вхождения в прибор. Для решения 
этой проблемы была предложена матрица Киргофа, в качестве регулязационного слагаемого.


\section*{слайд 9 Варианты вычисления не бинарной матрицы соседства}
Предлагалось рассмотреть следующие идеи для не бинарного отношения соседства.


\section*{слайд 10 Биннинг}
В рамках работы встречались распределения следующего вида. В самых популярных пакетах ROOT И RooUnfold, в качестве дискретизации 
предлагается только разделение на равные интервалы. При таком разбиении, решение задачи обратной свертки, не имеет особого смысла, 
тк мы получаем, однин бин, минимально будет модифицирован и множество несостоятельных бинов, которые из-за малого объема информации 
приведут у еще большему росту погрешности.

Для начала были приняты попытки к формализации данной задачи и точному ее решению. Я попытался представить как задачу максимизации 
следующего вида. Рассмотрим пары истинных значений в виде отрезков лежащих на оси X, а бины представим прямы паралеьных оси Y. 
Требуется расположить прямые таким образом чтобы количество пересечений было максимально, с условием некоторых ограничений, 
на возможное пересечение отрезка только один раз, и расстояния между бинами.

\section*{слайд 11 Биннинг эвристики}
Раннее у меня возникла идея использования следующих эвристик\dots

\end{document}