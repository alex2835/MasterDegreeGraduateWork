\documentclass[a4paper,12pt]{diplom}
% \usepackage[latin1]{inputenc}
% \usepackage[utf8]{inputenc}
\inputencoding{utf8} % Кодировка вашего файла


\usepackage{paratype} % Шрифты (можно отключить, если дает ошибку)
%% Немного увеличим шрифт в математическом режиме, чтобы соответствовать размерам Paratype-шрифтов
\DeclareMathSizes{12}{13.4}{11}{10}

\usepackage[left=3cm,right=2cm,top=2cm,bottom=2cm]{geometry} % Размеры полей
\usepackage[onehalfspacing]{setspace} % Полуторный интервал
%\renewcommand{\baselinestretch}{1.25} % Полуторный интервал
\usepackage{indentfirst} % Абзацный отступ в начале разделов
\setlength{\parindent}{1.25cm} % Величина абзацного отступа

\usepackage[pdftex]{graphicx} % Для вставки изображений
\usepackage{array} % Для таблиц
\usepackage{booktabs} % Для красивых таблиц 
\usepackage{tikz} % Рисунки с помощью TikZ
\usepackage[linesnumbered,lined,ruled]{algorithm2e} % Для оформления псевдокода
%\usepackage{algorithm} % Альтернатива оформления псевдокода
%\usepackage{algpseudocode} % Альтернатива оформления псевдокода
\usepackage{listings} % Оформление листингов программ
\usepackage{icomma} % Удаляем тонкий пробел после запятой в мат. режиме
\usepackage{centernot}
\usepackage[figure,table]{totalcount}

\begin{document}



\section*{слайд 2  Задача обратной свертки}

% Постановка основной задачи: Модификация и анализ возможностей методов восстановления обратной свертки распределений физических величин, 
% в том числе на основе расширенного SVD-алгоритма.

При проведении физических экспериментов, интерес может представлять распределение измеряемой физической величины. Однако, данные 
полученные в результате измерений, могут отличаться истинных значений из-за возможных внешних факторов или конечного разрешения 
аппаратуры. 

\section*{слайд 3 Гистограмный подход}
Математическая задача формулируется следующим образом. Имеется множество пар предполагаемых истинных и измеренных значений на основе которых 
строим два дискретных распределения, путем разбиения множества значений на интервалы и учета количества попавших элементов в каждый интервал.

Требуется алгоритм для преобразование измеренного распределения в предполагаемое истинное. Так формулируется задача обратной свертки.

Описание искажений определяется матрицей миграций, которая определят вероятность попадания случайный величины в какой-то бин
при условии что истинное значение попадает в свой бин. Это дает нам систему линейных отношений между смоделированным истинным и
измеренным распределениями

\section*{слайд 4 Матрица миграций}
Матрица миграций выглядит следующим образом


\section*{слайд 5 Задача минимизации}
Регуляризационный подход заключается в решение следующей задачи минимизации с добавлением слагаемого которое хранит дополнительную 
информацию об исходном распределении. 

\section*{слайд 6 Матрица соседства}
Для одномерного случая в оригинальной работе предлагается брать следующее слагаемое, которое является суммой квадратов вторых 
производных, и описаем гладкость искомого распределения.

\section*{слайд 7 Решение задачи минимизации}
Данная задача сводиться к переопределенной системе линейных уравнений. И решается с помощью сингулярного разложения матриц 
через последовательные замены.

\section*{слайд 8 Регуляризационное слагаемое для многомерного случая}
При решении задачи обратной свертки с космическими частицами столкнулись с проблемой, неспособности стандартного подхода решать задачу в многомерном 
случаи. Так как гладкость учитывается только односительно одного измерения.

В работе Юрия Викторовича была предложена следующая модификация, с использованием в качестве регуляризационного слагаемого матрицу Киргофа, 
которая в общем случаи описывает силу соседства. Эту работу только опубликовали и в ней рассматривалась матрица с бинарным отношением соседством.

И непосредственный интерес вызывало проверить ее определения отношения соседства небинарным образом.


\section*{слайд 9 Варианты вычисления не бинарной матрицы соседства}

Для вычисления силы связи предлагаются следующие идеи. 

1) Определять на основе площади соприкосновения, чем больше площадь тем сильнее связь 

2) На основе расстояния между центрами масс, в обратной пропорции.

3) И последняя идея. Вычислять основываясь на количестве истинных значений попавших в этот бин. То есть чем элементов из истинного спектра 
попали в данный бин, тем сильнее связь между ними.


\section*{слайд 10 Биннинг(Разделение на интервалы)}
В рамках работы встречались распределения следующего вида. В самых популярных пакетах ROOT И RooUnfold, в качестве дискретизации 
предлагается только разделение на равные интервалы. При таком разбиении, решение задачи обратной свертки, не имеет особого смысла. 
Тк при дополнительном разбиении первого бина можно улучшить результат. А во вторых большое количество не состоятельных бинов.  
Разница между тренировочной и тестовой выборкой может быть существенной, например, в последний бин попало 30 элементов в трен выборке,
а в тестовой 40, то есть разница в четверть, которая приведет к увеличению погрешности даже в сравнении с исходными данными.

\section*{слайд 11 Биннинг эвристики}

Для начала были приняты попытки к формализации данной задачи и точному ее решению. Я попытался представить как задачу максимизации 
следующего вида. Рассмотрим пары истинных значений в виде отрезков лежащих на оси X, а бины представим прямы параллельных оси Y. 
Требуется расположить прямые таким образом чтобы количество пересечений было максимально, с условием некоторых ограничений, 
на возможное пересечение отрезка только один раз, и расстояния между бинами.

Однако придуманные мной эвристики оказались более результативными.

Предлагается динамическое разбиение. То есть изначально рассматриваем один бин размером с все множество значений и последовательно 
делим самый крупный интервал по центру ли бо по медиане. Однако это ведет к еще одной проблеме слишком широких бинов, которые делают 
результат бесполезным. 

И последняя эвристика предполагает изначальное на равные интервалы и последующем динамическом разбиении.
Она показала лучший результат.


\section*{слайд 11 Результаты}
Удалось достигнуть следующих результатов. При автономном решении, только указании входных данных и количества бинов, удалось 
уменьшить среднеквадратическое отклонение для одномерного случая в 200 раз

И для двумерно случая в 4 раза.


\section*{слайд 11 Итог}

В результате работы были проверены идеи связанные вычислением небинарной матрицы соседства и автоматическом разбиении на интервалы.

Также было разработано приложение позволяющее решать задачу обратной свертки в многомерном случаи с реализацией вышеупомянутых идей.



\end{document}